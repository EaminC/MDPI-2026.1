\section{Introduction}

% !TeX root = ../main.tex

\subsection{Research Background}
In the design of modern information retrieval systems, the scale of retrievable items, the number of users, and the volume of interaction demands have grown rapidly, making \emph{accurate identification of user preferences} and \emph{fast, precise delivery of retrieval results} key optimization objectives. As early as the third century BCE, humans had already begun to design information retrieval systems for the ever-growing body of information. For instance, the first international library---the Library of Alexandria in ancient Greece---which collected the vast majority of books or copies available in the world at that time, had already developed a rudimentary system for information categorization and retrieval~\cite{1}. With the sharp increase in interaction scale and update frequency, such passive retrieval systems have gradually become insufficient to meet their original design goals. Therefore, to reduce users' search burden while still achieving efficient and accurate retrieval, R.~Armstrong~\cite{2} and colleagues first proposed an \emph{active} information retrieval system in 1995, which can be regarded as the early prototype of modern recommender systems. With the development of mobile Internet and the further rise in user--item interaction frequency, industry has an urgent demand for more efficient and more accurate recommendation technologies. For example, Meituan's 2024 financial report~\cite{3} states that its food delivery business generated 21.89 billion user--item interactions in the year, representing a year-on-year increase of 23.9\%. The growth in information volume and the rapid evolution of interaction categories also pose substantial and novel challenges for recommender system design.

To better cope with the computational load induced by interaction data overload, to more precisely infer user intent, and to generate recommendations that are both efficient and faithful to real user behaviors, researchers have increasingly focused on \emph{recommender systems}. The goal is to mine large-scale behavioral data---such as user ratings and interaction sequences---to learn representations of user preferences and to predict future interactions. For example, on movie platforms, one can predict the ratings that a user might give to unseen movies based on historical ratings, and then generate a ranked recommendation list~\cite{harper2015movielens} to help users discover preferred items more efficiently. In the digitization of urban life, interactions between users and POIs (Points of Interest)---e.g., residents' extensive behaviors related to food, clothing, housing, and transportation---can be leveraged to forecast future activities, thereby improving convenience in city living.

These applications heavily rely on the quality and quantity of historical interaction data, aiming to produce recommendation results that are close to users' true future behaviors. The abundance of interaction logs in the mobile Internet era provides a data foundation for behavior-driven predictive techniques. Meanwhile, with the rapid progress of natural language processing, recommendation approaches based on semantic understanding of content have received increasing attention in recent years. Motivated by this trend, this work proposes a hybrid recommender system framework that integrates content and behavior via large language models (LLMs), along with an implementation of the framework.

\subsection{Contributions and Paper Organization}
This work explores how to integrate LLM-based semantic reasoning with conventional recommendation pipelines that model user interactions and item content. The main contributions are:
\begin{itemize}
\item We propose an LLM-agent-based hybrid recommendation framework that combines semantic understanding with interaction modeling, supported by an explicit memory structure to represent and update user intent and context.
\item We develop a long--short-term adaptive interaction-based recommender that reweights historical interactions and uses LLM-assisted model scheduling to better handle preference shifts.
\item We design a content-based recommendation component with threshold filtering and regular-expression matching to reduce retrieval drift under semantic vectorization.
\item We introduce a behavior--content alignment strategy for hybrid-intent recommendation and validate the framework empirically on MovieLens.
\end{itemize}

The remainder of the manuscript is organized as follows. We review related work and foundational methods in the Related Work section. We then present the LLM-agent-based hybrid system design, followed by experimental evaluation. Finally, we conclude with a discussion of limitations and future directions.

\subsection{Research Challenges}
As discussed in the above survey, a variety of strategies have been developed for recommender systems, including interaction-based, content-based, and more recent approaches based on LLMs and hybrid combinations. Nevertheless, effectively integrating LLMs into existing recommender system frameworks remains a frontier problem. In practice, each paradigm has its limitations: interaction-based methods depend on abundant prior interactions and suffer from cold start; content-based methods are constrained by feature extraction quality and generalization; LLM-based recommenders are limited by compute requirements and remote inference costs, making it difficult for them to serve as the sole backbone of a recommender system. These observations motivate a balanced architecture that incorporates LLMs, integrates the strengths of existing systems, and enables LLMs to contribute at critical points with reasonable overhead.

To this end, this work designs an LLM-based hybrid recommender framework and optimizes different components using LLMs. The key challenges can be summarized as follows:

First, interaction-based models lack personalized corrections with respect to the recommendation objects. Users exhibit both long-term and short-term interest shifts, while model invocation is typically uniform, which can cause substantial errors for certain users. Since interaction-based models do not explicitly understand content, identifying interest changes incurs additional overhead. Given the massive scale of interaction matrices in modern recommender systems, this can lead to significant computational costs.

Second, content-based recommender systems can better extract user and item features, but without explicit behavioral signals, they may fail to respond to short-lived spikes in large-scale interactions triggered by trending events, leading to recommendation bias. Moreover, if the recall set is too large, per-query computation increases; since fine-grained ranking is ultimately required, substantial unnecessary computation may be wasted, degrading overall system performance.

Third, the role and paradigm of LLMs in recommender workflows remain under-specified. Recommender systems are large engineering systems that implement end-to-end information retrieval functionalities. After incorporating LLMs, one must maximize their benefits in semantic understanding and feature extraction while avoiding significant increases in compute overhead and system complexity, and also align interaction-based and content-based subsystems. This requires a more general and cost-performance balanced framework adaptable across datasets.

\subsection{Research Methodology}
Building on prior work (reviewed in the Related Work section) and the challenges identified above, this work proposes an LLM-based hybrid recommender system framework that integrates behavior-based and content-based recommender systems, strengthens semantic understanding and feature extraction, and improves overall interaction prediction. The methodology includes the following components:

First, we design a long-/short-term interest change detection and adaptive recommendation module based on user interactions and LLM agents. On top of classical matrix factorization, we add a linear layer with adjustable weights to manually control the attention range, enabling adaptation to different long-/short-term recommendation scenarios. We further design an agent memory update mechanism to more accurately detect interest changes, quantify their magnitude to some extent, and make model orchestration decisions, thereby addressing the lack of content understanding and the high cost of frequent model updates in traditional behavior-based recommenders.

Second, building on the GPT4Rec framework proposed by Li et~al.~\cite{li2023gpt4rec}, we correct certain vectorization biases and similarity estimation errors caused by overly long and noisy content. We then propose a regularization-threshold-based Query--Search framework for feature enhancement in content-based recommendation. Leveraging LLMs' semantic understanding and generalization, this module predicts future query terms and generates coarse-ranking results efficiently for downstream refinement, reducing wasted computation in traditional pipelines.

Third, we propose a complete LLM-based hybrid alignment framework. By anchoring the recalled results from the content-based recommender and using LLM-driven semantic understanding together with an interaction-based long-/short-term adaptive network for correction, the final results can jointly reflect behavioral signals and content understanding while reducing LLM resource overhead. This framework provides a practical approach to integrating LLMs into existing recommender systems and offers strong extensibility and generalization potential for future work.
