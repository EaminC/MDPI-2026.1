\section{Conclusions and Future Work}

\subsection{Summary}
Motivated by recent advances in large language models (LLMs) and by the complementary strengths of content-based and behavior-based recommender systems, this work proposes a hybrid recommendation framework that integrates semantic understanding with interaction modeling. Specifically, we develop: (i) a long--short-term adaptive recommender that combines interaction reweighting with an LLM-agent-based scheduling mechanism; (ii) a content-based recommender enhanced by threshold filtering and regular-expression matching to mitigate retrieval drift; and (iii) an LLM-driven hybrid-intent recommendation framework that aligns and corrects content recall using interaction signals. To support semantic reasoning and interpretability, we also introduce an agent memory structure to represent and update user intent and context, and provide an end-to-end system design that integrates these components.

We further conduct experiments to evaluate the proposed framework. The results suggest that hybrid-intent recommendation with LLM-assisted alignment can improve ranking-oriented metrics and enhance overall interpretability and semantic reasoning capabilities. As an early exploration of integrating LLMs with traditional recommender algorithms, this work offers a practical design paradigm and empirical evidence that may support subsequent research and real-world applications.

\subsection{Future Work}
Future work will focus on interactive recommendation in realistic settings and on recommendation techniques that more deeply leverage semantic understanding and reasoning. In particular, most offline metrics are computed on historical logged data, whereas recommendation itself can influence user behavior through exposure effects. Building on LLMs and other semantic technologies, it is promising to incorporate richer user interaction loops and to optimize recommendation models using real-time datasets from production environments. Such settings may also motivate evaluation protocols beyond Recall and NDCG that better reflect causal effects, user satisfaction, and long-term utility.

In addition, LLM-based recommenders are well-positioned to handle more complex and mixed-context scenarios. For example, as discussed in the urban-space setting, POI recommendation includes rich spatiotemporal signals beyond typical online items, and thus requires stronger capabilities in language understanding, semantic grounding, and spatiotemporal reasoning---which remain challenging for existing systems. Further research on LLMs and agentic architectures may enable more effective and valuable solutions for city-scale recommendation problems.
